\documentclass[11pt]{article}

\usepackage[margin=1in]{geometry}
\usepackage{amsmath, amsfonts, amssymb}
\usepackage[none]{hyphenat}
\usepackage{fancyhdr}
\usepackage{graphicx}
\usepackage{float}

\pagestyle{fancy}
\fancyhead{}
\fancyfoot{}
\fancyhead[L]{\slshape\MakeUppercase{Place Title Here}}
\fancyhead[R]{\slshape Student Name}
\fancyfoot[C]{\thepage}
%\renewcommand{\headrulewidth}{0pt}
\renewcommand{\footrulewidth}{0pt}

\begin{document}

\begin{titlepage}
\begin{center}
\vspace*{1cm}
\Large{\textbf{IB Mathematics VS}}\\
\Large{\textbf{Internal Assessment}}
\vfill
\line(1,0){400}\\[1mm]
\huge{\textbf{This is a Sample Title}}\\[3mm]
\Large{\textbf{-This is a Sample Subtitle-}}\\
\line(1,0){400}\\
\vfill
By Student Name\\
Candidate\\
\today\\
\end{center}
\end{titlepage}

\section{Introduction}
Differential Equations and their importance. Frobenius method and its applications to differential equations. Legendre, Bessel, Hermite and Laguerre Differential Equations. Properties of Legendre Polynomials: Rodrigues Formula, Generating Function, Orthogonality. Simple recurrence relations.Expansion of function in a series of Legendre Polynomials.

\section{Scoring Criteria}
Introduction to Scilab, Advantages and disadvantages, Scilab environment, Command window, Figure window, Edit window, Variables and arrays, Initialising variables in Scilab, Multidimensional arrays, Sub-array, Special values, Displaying output data, data file, Scalar and array operations, Hierarchy of operations, Built in Scilab functions, Introduction to plotting, 2D and 3D plotting, Branching Statements and program design, Relational and logical operators, the while loop, for loop, details of loop operations, break and continue statements, nested loops, logical arrays and vectorization. User defined functions, Introduction to Scilab functions, Variable passing in Scilab, optional arguments, preserving data between calls to a function, Complex and Character data, string function, Multidimensional arrays an introduction to Scilab file processing, file opening and closing, Binary I/o functions, comparing binary and formatted functions, Numerical methods and developing the skills of writing a program.

\section{Communication}


\section{Mathematical Presentation}
Thermodynamic Potentials: Internal Energy, Enthalpy, Helmholtz Free Energy, Free Energy. Their Definitions, Properties and Applications. Magnetic Work, Cooling due to adiabatic demagnetization, First and second order Phase Transitions with examples, Clausius Clapeyron Equation and Ehrenfest equations

\section{Personal Engagement}
Distribution of Velocities: Maxwell-Boltzmann Law of Distribution of Velocities in an Ideal Gas and its Experimental Verification. Mean, RMS and Most Probable Speeds. Degrees of Freedom. Law of Equipartition of Energy (No proof required). Specific heats of Gases.

\section{Reflection}
Mean Free Path. Collision Probability. Estimation of Mean Free Path. Transport Phenomenon in Ideal Gases: (1) Viscosity, (2) Thermal Conductivity and (3) Diffusion. Brownian Motion and its Significance.

\section{Use of Mathematics}
This is one of the core papers in physics curriculum which introduces the concept of Boolean algebra and the basic digital electronics. In this course, students will be able to understand the working principle of CRO, Data processing circuits, Arithmetic Circuits, sequential circuits like registers, counters etc. based on flip flops. In addition, students will get an overview of microprocessor architecture and programming

\section{Conclusion}
Difference between Analog and Digital Circuits.Examples of linear and digital ICs, Binary Numbers. Decimal to Binary and Binary to Decimal Conversion.BCD, Octal and Hexadecimal numbers. AND, OR and NOT Gates (realization using Diodes and Transistor). NAND and NOR Gates as Universal Gates. XOR and XNOR Gates and application as Parity Checkers.

\section{Using \LaTeX}
$\sqrt{x^{2+\sqrt{y^{\sqrt{z^2}}}}}$

\end{document}